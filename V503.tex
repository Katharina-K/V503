\input{header.tex}

\subject{V503}
\title{Der Millikan-Öltröpfchenversuch}
\date{
  Durchführung: 13.12.2022
  \hspace{3em}
  Abgabe: 20.12.2022
}
\author{Katharina Kürschner \and Leonard Trinschek}

\begin{document}

\maketitle
\thispagestyle{empty}
\tableofcontents
\newpage

\section{Ziel}
Ziel dieses Versuches ist die Bestimmung der 
Elementarladung mithilfe einer leicht abgewandelten 
Version der Millikan Methode.
\label{sec:Ziel}




\section{Theorie}
\label{sec:Theorie}

Die Milikan-Methode zur Bestimmung der Elementarladung basiert auf der Zerstäubung von Öltröpfchen in das elektrische Feld eines Plattenkondensators. 
Durch die Reibung der Tröpfchen mit der Luft werden sie elektrisch geladen. Die Ladung $q$ der Tröpfchen kann nur ein ganzzahliges Vielfaches der 
Elementarladung sein. Das elektrische Feld des Plattenkondensators ist vertikal ausgerichtet, wodurch die auf die geladenen Teilchen wirkende 
elektrische Kraft $\vec{F}_\text{el}$ parallel oder antiparallel zur Gravitationskraft $\vec{F}_\text{g}$ wirkt. Zusätzlich wirkt die Stokesche 
Reibungskraft $\vec{F}_\text{R}$ entgegen der Bewegungsrichtung, da sich die Teilchen mit einer Geschwindigkeit $\vec{v}$ durch den luftgefüllten 
Raum bewegen.

Die Wirkung dieser Kräfte auf ein Teilchen kann durch folgende Gleichungen beschrieben werden:

\begin{align}
    \label{eqn:Kraefte}
    \vec{F}_\text{g} &= m \vec{g} \\
    \vec{F}_\text{el} &= q \vec{E} \\
    \vec{F}_\text{R} &= -6\symup{\pi}r\eta_\text{L} \vec{v}
\end{align}

Hierbei steht $m$ für die Masse des Teilchens, $\vec{g}$ für die Fallbeschleunigung, $\eta_\text{L}$ für die Viskosität der Luft und $r$ für den Radius 
des Teilchens.

Nach einer kurzen Zeit stellt sich ein Kräftegleichgewicht ein, bei dem sich die Tröpfchen mit konstanter Geschwindigkeit bewegen. Bei abgeschaltetem 
elektrischen Feld bewegen sich die Öltröpfchen mit der Geschwindigkeit $v_0$ und erhalten durch den Auftrieb der Luft den Radius:

\begin{equation}
    \label{eqn:Radius}
    r = \sqrt{\frac{9 \eta_\text{L}(v_\text{ab} - v_\text{auf})}{4g(\rho_\text{Oel}- \rho_\text{L})}}.
\end{equation}

Das Kräftegleichgewicht führt zu folgender Gleichung:

\begin{equation*}
    \frac{4\symup{\pi}}{3}r^3(\rho_\text{Oel}- \rho_\text{L})g = 6 \symup{\pi} \eta_\text{L}r v_0.
\end{equation*}

Abhängig von der Polung des elektrischen Feldes wirken die elektrostatische Kraft und die Reibungskraft in verschiedene Richtungen. Die Orientierung 
der Kräfte kann der Abbildung \ref{fig:Kraeftegleichgewicht} entnommen werden.

\begin{figure}
    \centering
    \includegraphics[width = .9\textwidth]{Bilder/Kraeftegleichgewicht.png}
    \caption{Orientierung der wirkenden Kräfte bei unterschiedlicher Polung des elektrischen Feldes. \cite{1}}
    \label{fig:Kraeftegleichgewicht}
\end{figure}

Wenn die obere Platte des Kondensators positiv geladen ist und eine ausreichend große Spannung anliegt, bewegt sich das Öltröpfchen mit der 
Geschwindigkeit $v_\text{auf}$ nach oben. Das Kräftegleichgewicht ergibt sich zu:

\begin{equation*}
    \label{eqn:Kraefte_1}
    \frac{4\symup{\pi}}{3}r^3(\rho_\text{Oel} + \rho_\text{L})g + 6 \symup{\pi} \eta_\text{L}r v_\text{auf} = qE.
\end{equation*}

Bei entgegengesetzter Polung des elektrischen Feldes ergibt sich:

\begin{equation*}
    \label{eqn:Kraefte_2}
    \frac{4\symup{\pi}}{3}r^3(\rho_\text{Oel}- \rho_\text{L})g - 6 \symup{\pi} \eta_\text{L}r v_{\text{ab}} = -qE,
\end{equation*}

wobei $v_{\text{ab}}$ die nach unten gerichtete Geschwindigkeit ist.

Aus diesen beiden Gleichungen kann die Ladung $q$ des Öltröpfchens bestimmt werden:

\begin{equation}
    \label{eqn:q}
    q = \frac{9}{2} \symup{\pi} \sqrt{\frac{\eta_\text{L}^3(v_\text{ab} - v_\text{auf})}{g(\rho_\text{Oel}- \rho_\text{L})}} \cdot \frac{v_\text{ab} + v_\text{auf}}{E},
\end{equation}

wobei $E$ den Betrag des elektrischen Feldes darstellt. Die Geschwindigkeiten sind durch folgenden Zusammenhang gegeben:

\begin{equation}
    \label{eqn:v_0}
    2v_0 = v_\text{ab} - v_\text{auf}.
\end{equation}

Bei diesen Gleichungen \ref{eqn:q} muss eine Korrektur durchgeführt werden, weil die Gleichungen nur für Tröpfchen gelten deren Abmessungen größer 
als die mittlere freie Weglänge in Luft ist.
Die Korrektur ist dabei gegeben als
\begin{align}
\label{eq:Theorie_Cunningham_Viskositaet}
\eta_\text{eff}=\eta_\text{L}\left( \frac{1}{1+B\frac{1}{pr}} \right),
\end{align}
sie wird als \textbf{Cunningham-Korrekturterm} bezeichnet. 
Dazu wird der Luftdruck $p$ und die experimentell bestimmbare Konstante $B =  \num{6.17e-3}\, \text{Torr}\cdot\unit{\centi\metre}$ \cite{1} verwendet.
Es gilt $1\,\text{Torr} \approx \qty{133.322}{\pascal}$ \cite{2}.
Für die korrigierte Ladung gilt
\begin{equation}
    \label{eqn:q_korrigiert}
    q_\text{real} = q_0 \left(1+ \frac{B}{pr}\right)^{3/2}.
\end{equation}
\section{Versuchsaufbau und Durchführung}
\label{sec:Durchführung}

\subsection{Versuchsaufbau}
\label{subsec:Versuchsaufbau}
Der Versuchsaufbau besteht aus einer Kammer mit einem Plattenkondensator, der eine kleine Öffnung an der Oberseite aufweist. 
Diese Oberseite wird zum Einspritzen von zerstäubten Öltröpfchen verwendet. Die Platten des Kondensators haben einen Abstand 
von $d = (7,6250 \pm 0,0051) \, \si{\milli\meter}$.

Um die Tröpfchen gut sichtbar zu machen, werden sie seitlich von einer Halogenlampe beleuchtet. Die Temperatur der Luft in der 
Kammer wird mit einem Thermowiderstand kontrolliert, dessen Wert an einem Multimeter abgelesen werden kann. Ebenso kann die 
Spannung zwischen den beiden Kondensatorplatten an einem Multimeter abgelesen werden.

Durch das Zerstäuben sind die meisten Öltröpfchen geladen, während einige nicht geladen sind. Die nicht geladenen Tröpfchen 
können durch ein schwach radioaktives $\alpha$-Präparat ionisiert werden. Durch einen Schalter kann das Präparat abgeschirmt 
oder "aktiviert" werden.

Die Polung der Kondensatorplatten kann mit einem Schalter geändert werden. Mit einer Libelle kann überprüft und eingestellt 
werden, ob die Apparatur gerade steht. Die Tröpfchen können mit einem Mikroskop beobachtet werden.

Der Versuchsaufbau ist in Abbildung \ref{fig:Versuchsaufbau} dargestellt.

\begin{figure}[H]
    \centering
    \caption{Schematischer Aufbau der Versuchsappartatur zum Millikan-Öltröpchen-Versuch.\cite{1}}
    \includegraphics[width=\textwidth]{Bilder/Versuchsaufbau.png}
    \label{fig:Versuchsaufbau}
\end{figure}


\subsection{Durchführung}
\label{subsec:Durchführung}

Zu Beginn wird die Ausrichtung der Apparatur überprüft, um sicherzustellen, dass sie waagerecht steht. Dies wird mithilfe einer 
Nadel durchgeführt. Anschließend werden die Kondensatorplatten geerdet und Öltröpfchen in die Kammer eingesprüht. Während des 
Einsprühens wird mithilfe eines Mikroskops überwacht, wie viele Tröpfchen in die Kammer gelangen.

Nun werden bei zwei verschiedenen Spannungen 22 verschiedene Tröpfchen beobachtet. Dabei werden die Zeiten für den Aufstieg 
und den Abstieg eines Tröpfchens über eine festgelegte Strecke jeweils drei Mal gemessen. Vor jeder Messung wird der Wert 
der Temperatur für jedes Tröpfchen erfasst.

Die verwendeten Spannungen betragen 200 V und 230 V. Durch das Umpolen mithilfe des Schalters werden die Tröpfchen entweder 
in Aufwärts- oder Abwärtsbewegung gebracht.



\input{Inhalt/fehlerrechnung.tex}
\section{Auswertung}
\label{sec:Auswertung}

\subsection{Ünerprüfen der Messwerte im Rahmen der Messgenauigkeit}
In \ref{tab:} sind die Messwerte für diesen Versuch aufgeführt. Dabei handelt es sich um die Spannung $U$, 
die Steigzeit $t_{\text{auf}}$, die Fallzeit $t_{\text{ab}}$ und die Temperatur $T$. Da für jeden Öltropfen drei Steig- als auch 
Fallzeiten gemessen wurden, wurde der Mittelwert aus diesen berechnet. Andernfalls entspricht der Mittelwert dem jeweiligen 
Einzelmesswert.

Tabelle einfügen U, v0, tauf alle, tab alle, Steigzeit mittel, Fallzeit mittel, T

In \ref{tab:} sind die Ergebnisse der Messungen aufgeführt. Aus den Zeiten $t_{\text{auf}}$ und $t_{\text{ab}}$ 
wurde die entsprechende Geschwindigkeit $v_{\text{auf}}$ und $v_{\text{ab}}$ über die verwendete Messstrecke $s = \SI{0.5}{\mm}$
mit $v=\frac{s}{t}$ berechnet. Die Luftviskosität $\eta_{L}$ wurde gemäß Abbildung 3 in \cite{1} verwendet. Zudem wurde noch die  
Differenz der Geschwindigkeiten $v_{\text{ab}} - v_{\text{auf}}$ bestimmt.

Tabelle einfügen vauf, vab, vab-vauf, v0, nL

Um sich auf einen Bereich einzuschränken, werden alle Werte, die die Relation 
\begin{equation*}
    0.5\leq \frac{2v_0}{\bar{v}_{\text{auf}}-\bar{v}_{\text{ab}}}\leq 1.5
\end{equation*}
erfüllen, als auswertbar angenommen.

\subsection{Bestimmung der Ladung und der Radien der Öltröpfchen}

\subsection{Bestimmung der Elementarladung}

\subsection{Bestimmung der Avogadrokonstante}




\section{Diskussion}
\label{sec:Diskussion}
Es werden Theoriewerte von $N_A = 6.0221 \cdot 10^{23} \frac{1}{\text{mol}}$ \cite{3} und $e_t=1.6022 \cdot 10^{-19} C$ angenommen \cite{4}.
Beim Vergleich der beiden berechneten Elementarladungen miteinander 
fällt auf, dass diese nur marginal voneinaner abweichen.
Sowogl der unkorrigierte Wert von $e_0=(1.4728 \pm 0.0821) 10^{-19} C$, als auch 
der korrigierte Wert von
$e_{0k}=(1.4741 \pm 0.0829) 10^{-19}C$ weichen nach \ref{sec:Fehlerrechnung}
um 8.07 \% vom Theoriewert ab. Demnach ist auch die Abweichung für
die Avogadrokonstaten für die unkorrigierte und korrigierte Ladung
identisch und beträgt 8.7 \%. 
Insgesamt war es also möglich die Elementarladung und somit auch die
Avogadrokonstante recht genau zu bestimmen. Die genaue Bestimmung wird hauptsächlich
durch das vorige Aussortieren ungeeigneter Messwerte ermöglicht. 
Wahrscheinlich wäre eine noch genauere Bestimmung möglich, wenn die 
Empfehlung, nur langsame Öltröpfchen zu messen, beachtet worden wäre. 





\newpage
% \printbibliography{}
% \nocite{matplotlib}
% \nocite{numpy}
% \nocite{scipy}
% \nocite{uncertainties}
% \nocite{reback2020pandas}

\begin{thebibliography}{9}
  \bibitem{1} Physikalisches Anfängerpraktikum der TU Dortmund: Versuch V503 -   Der Millikan-Öltröpfchenversuch. Stand: Mai 2023.
  \bibitem{2} Entnommen aus: https://www.chemie.de/lexikon/Torr.html . Stand: Mai 2023.
  \bibitem{3} Entnommen aus: https://www.chemie.de/lexikon/Avogadro-Konstante.html . Stand: Mai 2023.
  \bibitem{4} Entnommen aus: https://www.chemie.de/lexikon/Elementarladung.html . Stand: Mai 2023.
\end{thebibliography}
\section{Anhang}

\begin{figure}[H]
    \centering
    \includegraphics[width=\textwidth]{Bilder/V503.1.jpeg}
\end{figure}

\begin{figure}[H]
    \centering
    \includegraphics[width=\textwidth]{Bilder/V503.2.jpeg}
\end{figure}

\begin{figure}[H]
    \centering
    \includegraphics[width=\textwidth]{Bilder/V503.3.jpeg}
\end{figure}

\begin{figure}[H]
    \centering
    \includegraphics[width=\textwidth]{Bilder/V503.4.jpeg}
\end{figure}

\begin{figure}[H]
    \centering
    \includegraphics[width=\textwidth]{Bilder/V503.5.jpeg}
\end{figure}

\begin{figure}[H]
    \centering
    \includegraphics[width=\textwidth]{Bilder/V503.6.jpeg}
\end{figure}
\end{document}
