\section{Diskussion}
\label{sec:Diskussion}
Es werden Theoriewerte von $N_A = 6.0221 \cdot 10^{23} \frac{1}{\text{mol}}$ \cite{3} und $e_t=1.6022 \cdot 10^{-19} C$ angenommen \cite{4}.
Beim Vergleich der beiden berechneten Elementarladungen miteinander 
fällt auf, dass diese nur marginal voneinaner abweichen.
Sowogl der unkorrigierte Wert von $e_0=(1.4728 \pm 0.0821) 10^{-19} C$, als auch 
der korrigierte Wert von
$e_{0k}=(1.4741 \pm 0.0829) 10^{-19}C$ weichen nach \ref{sec:Fehlerrechnung}
um 8.07 \% vom Theoriewert ab. Demnach ist auch die Abweichung für
die Avogadrokonstaten für die unkorrigierte und korrigierte Ladung
identisch und beträgt 8.7 \%. 
Insgesamt war es also möglich die Elementarladung und somit auch die
Avogadrokonstante recht genau zu bestimmen. Die genaue Bestimmung wird hauptsächlich
durch das vorige Aussortieren ungeeigneter Messwerte ermöglicht. 
Wahrscheinlich wäre eine noch genauere Bestimmung möglich, wenn die 
Empfehlung, nur langsame Öltröpfchen zu messen, beachtet worden wäre. 



