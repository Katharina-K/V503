\section{Auswertung}
\label{sec:Auswertung}

\subsection{Ünerprüfen der Messwerte im Rahmen der Messgenauigkeit}
In \ref{tab:} sind die Messwerte für diesen Versuch aufgeführt. Dabei handelt es sich um die Spannung $U$, 
die Steigzeit $t_{\text{auf}}$, die Fallzeit $t_{\text{ab}}$ und die Temperatur $T$. Da für jeden Öltropfen drei Steig- als auch 
Fallzeiten gemessen wurden, wurde der Mittelwert aus diesen berechnet. Andernfalls entspricht der Mittelwert dem jeweiligen 
Einzelmesswert.

Tabelle einfügen U, v0, tauf alle, tab alle, Steigzeit mittel, Fallzeit mittel, T

In \ref{tab:} sind die Ergebnisse der Messungen aufgeführt. Aus den Zeiten $t_{\text{auf}}$ und $t_{\text{ab}}$ 
wurde die entsprechende Geschwindigkeit $v_{\text{auf}}$ und $v_{\text{ab}}$ über die verwendete Messstrecke $s = \SI{0.5}{\mm}$
mit $v=\frac{s}{t}$ berechnet. Die Luftviskosität $\eta_{L}$ wurde gemäß Abbildung 3 in \cite{1} verwendet. Zudem wurde noch die  
Differenz der Geschwindigkeiten $v_{\text{ab}} - v_{\text{auf}}$ bestimmt.

Tabelle einfügen vauf, vab, vab-vauf, v0, nL

Um sich auf einen Bereich einzuschränken, werden alle Werte, die die Relation 
\begin{equation*}
    0.5\leq \frac{2v_0}{\bar{v}_{\text{auf}}-\bar{v}_{\text{ab}}}\leq 1.5
\end{equation*}
erfüllen, als auswertbar angenommen.

\subsection{Bestimmung der Ladung und der Radien der Öltröpfchen}

\subsection{Bestimmung der Elementarladung}

\subsection{Bestimmung der Avogadrokonstante}



