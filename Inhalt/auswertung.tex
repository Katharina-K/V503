\section{Auswertung}
\label{sec:Auswertung}

\subsection{Überprüfen der Messwerte im Rahmen der Messgenauigkeit}
In \ref{tab:Messwerte} sind die Messwerte für diesen Versuch aufgeführt. Dabei handelt es sich um die Spannung $U$, 
die Steigzeit $t_{\text{auf}}$, die Fallzeit $t_{\text{ab}}$ und die Temperatur $T$. Da für jeden Öltropfen drei Steig- als auch 
Fallzeiten gemessen wurden, wurde der Mittelwert aus diesen berechnet. Andernfalls entspricht der Mittelwert dem jeweiligen 
Einzelmesswert.

\begin{landscape}
\begin{table}[!ht]
    \centering
    \begin{tabular}{c c c c c c c c c c c c}
        \toprule
		Tröpf- & Spannung & $t_0$ & Steigzeit 1 & Steigzeit 2 & Steigzeit 3 & Fallzeit 1 & Fallzeit 2 & Fallzeit 3 & Steigzeit & Fallzeit & Temperatur\\
        chen & & & & & & & & & Mittel & Mittel& \\
         & $U$ [\si{\volt}]&  & $t_{1,\text{auf}}$ [\si{\second}] & $t_{2,\text{auf}}$ [\si{\second}]& $t_{3,\text{auf}}$ [\si{\second}] & $t_{1,\text{ab}}$ [\si{\second}] & $t_{2,\text{ab}}$ [\si{\second}] & $t_{2,\text{ab}}$ [\si{\second}] & $\overline{t_{\text{auf}}}$ [\si{\second}] & $\overline{t_{\text{ab}}}$ [\si{\second}] & $T$ [\si{\celsius}]\\
        \midrule
            1&        201 & 21,11 &                      4,76 &                      4,23 &                      4,47 &                3,13 &                3,15 &                3,22 &                                               4,49 &       3,17 &                 22 \\
            2&        201 & 24,52 &                      6,84 &                      6,91 &                       5,6 &                4,54 &                4,38 &                4,13 &                                               6,45 &       4,35 &                 22 \\
            3&        201 & 37,21 &                       4,7 &                      3,13 &                      3,33 &                2,66 &                2,61 &                2,59 &                                               3,72 &       2,62 &                 22 \\
            4&        201 & 32,45 &                      3,24 &                      2,62 &                      2,72 &                2,59 &                 2,2 &                 2,5 &                                               2,86 &       2,43 &                 22 \\
            5&        201 & 18,24 &                      4,72 &                      5,56 &                      5,77 &                 3,2 &                3,67 &                3,52 &                                               5,35 &       3,46 &                 22 \\
            6&        201 & 26,39 &                      3,07 &                      3,27 &                      3,16 &                2,52 &                2,72 &                 2,7 &                                               3,17 &       2,65 &                 22 \\
            7&        201 & 29,28 &                      2,41 &                      2,44 &                      2,54 &                2,39 &                 2,4 &                2,38 &                                               2,46 &       2,39 &                 22 \\
            8&        201 & 45,39 &                       2,7 &                      2,94 &                      2,86 &                2,62 &                2,58 &                2,43 &                                               2,83 &       2,54 &                 22 \\
            9&        230 & 22,45 &                       3,6 &                      3,72 &                      4,09 &                3,87 &                3,78 &                 3,7 &                                               3,80 &       3,78 &                 21 \\
            10&        230 & 31,59 &                      3,66 &                      2,91 &                      3,01 &                2,52 &                2,71 &                2,58 &                                               3,19 &       2,60 &                 21 \\
            11&        230 & 19,01 &                      2,93 &                      2,86 &                      2,91 &                2,05 &                2,41 &                2,26 &                                               2,90 &       2,24 &                 21 \\
            12&        230 & 53,22 &                      3,68 &                      3,61 &                      3,72 &                2,83 &                3,29 &                3,24 &                                               3,67 &       3,12 &                 21 \\
            13&        230 & 29,45 &                      6,23 &                      7,02 &                      7,49 &                5,58 &                6,32 &                5,47 &                                               6,91 &       5,79 &                 21 \\
            14&        230 & 54,16 &                      2,57 &                      2,77 &                      2,93 &                 2,6 &                2,88 &                2,61 &                                               2,76 &       2,70 &                 21 \\
            15&        230 & 24,43 &                      5,03 &                      5,02 &                      5,31 &                 3,7 &                3,58 &                3,58 &                                               5,12 &       3,62 &                 21 \\
            16&        230 & 19,17 &                      2,97 &                      3,15 &                      3,17 &                3,06 &                2,69 &                2,82 &                                               3,10 &       2,86 &                 21 \\
            17&        230 & 29,66 &                      3,82 &                      3,66 &                      3,82 &                2,89 &                2,58 &                2,97 &                                               3,77 &       2,81 &                 21 \\
            18&        230 & 29,08 &                       2,5 &                      2,48 &                      2,43 &                1,97 &                2,24 &                 2,4 &                                               2,47 &       2,20 &                 21 \\
            19&        230 & 29,05 &                      2,86 &                      3,51 &                      2,96 &                2,37 &                2,62 &                2,96 &                                               3,11 &       2,65 &                 21 \\
            20&        230 & 32,07 &                      2,21 &                      2,33 &                       2,4 &                1,95 &                2,04 &                2,14 &                                               2,31 &       2,04 &                 21 \\
            21&        230 & 50,09 &                      2,53 &                      2,68 &                      2,75 &                2,11 &                 2,1 &                2,25 &                                               2,65 &       2,15 &                 21 \\
            22&        230 & 42,67 &                      2,17 &                      2,12 &                      2,05 &                1,99 &                2,38 &                2,28 &                                               2,11 &       2,22 &                 21 \\
        \bottomrule
        \end{tabular}
        \caption{Die aufgenommenen Messwerte.}
        \label{tab:Messwerte}
    \end{table}
\end{landscape}

In \ref{tab:Tabelle2} sind die Ergebnisse der Messungen aufgeführt. Aus den Zeiten $t_{\text{auf}}$ und $t_{\text{ab}}$ 
wurde die entsprechende Geschwindigkeit $v_{\text{auf}}$ und $v_{\text{ab}}$ über die verwendete Messstrecke $s = \SI{0.5}{\mm}$
mit $v=\frac{s}{t}$ berechnet. Die Luftviskosität $\eta_{L}$ wurde gemäß Abbildung 3 in \cite{1} verwendet. Zudem wurde noch die  
Differenz der Geschwindigkeiten $v_{\text{ab}} - v_{\text{auf}}$ bestimmt.

\begin{table}[!ht]
    \centering
    \begin{tabular}{c c c c c c}
        \toprule
        Tröpfchen &Steig- & Fall- & Differenz- & $v_0$ & Luftviskosität\\
         & geschwindigkeit & geschwindigkeit & geschwindigkeit & & \\
         & $v_{\text{auf}}$ [\si{\milli\meter\per\second}] & $v_{\text{ab}}$ [\si{\milli\meter\per\second}] & $v_{\text{ab}} - v_{\text{auf}}$ [\si{\milli\meter\per\second}] &  & $\eta_{L}$ [\si{\micro\newton\second\per\square\meter}]\\
        \midrule
         1&   0,11 &                  0,16 &                      0,05 &  0,02 &          18,34  \\
         2&   0,08 &                  0,11 &                      0,04 &  0,02 &          18,34  \\
         3&   0,13 &                  0,19 &                      0,06 &  0,01 &          18,34  \\
         4&   0,17 &                  0,21 &                      0,03 &  0,02 &          18,34  \\
         5&   0,09 &                  0,14 &                      0,05 &  0,03 &          18,34  \\
         6&   0,16 &                  0,19 &                      0,03 &  0,02 &          18,34  \\
         7&   0,20 &                  0,21 &                      0,01 &  0,02 &          18,34  \\
         8&   0,18 &                  0,20 &                      0,02 &  0,01 &          18,34  \\
         9&   0,13 &                  0,13 &                      0,00 &  0,02 &          18,28  \\
         10&   0,16 &                  0,19 &                      0,04 &  0,02 &          18,28  \\
         11&   0,17 &                  0,22 &                      0,05 &  0,03 &          18,28  \\
         12&   0,14 &                  0,16 &                      0,02 &  0,01 &          18,28  \\
         13&   0,07 &                  0,09 &                      0,01 &  0,02 &          18,28  \\
         14&   0,18 &                  0,19 &                      0,00 &  0,01 &          18,28  \\
         15&   0,10 &                  0,14 &                      0,04 &  0,02 &          18,28  \\
         16&  0,16 &                  0,17 &                      0,01 &  0,03 &          18,28  \\
         17&   0,13 &                  0,18 &                      0,05 &  0,02 &          18,28  \\
         18&   0,20 &                  0,23 &                      0,02 &  0,02 &          18,28  \\
         19&   0,16 &                  0,19 &                      0,03 &  0,02 &          18,28  \\
         20&   0,22 &                  0,25 &                      0,03 &  0,02 &          18,28  \\
         21&   0,19 &                  0,23 &                      0,04 &  0,01 &          18,28  \\
         22&   0,24 &                  0,23 &                     -0,01 &  0,01 &          18,28  \\
        \bottomrule
    \end{tabular}
    \caption{Aus den Messwerten berechnete Steig- und Fallgeschwindigkeiten, deren Differenz, 
	sowie $v_0$ und unkorrigierte Viskosität der Luft.}
    \label{tab:Tabelle2}
\end{table}

Um sich auf einen Bereich einzuschränken, werden alle Werte, die die Relation 
\begin{equation*}
    0.75\leq \frac{2v_0}{\bar{v}_{\text{auf}}-\bar{v}_{\text{ab}}}\leq 1.25
\end{equation*}
erfüllen, als auswertbar angenommen. Die Ergebnisse zur Gültigkeit der Öltröpfchen ist in \autoref{tab:Gültigkeitsbereich} 
zu finden.

\begin{table}[!ht]
    \centering
    \begin{tabular}{c c c c c c}
        \toprule
        Tröpfchen & Differenzgeschwindigkeit  & $v_0$   & $\frac{2v_0}{\bar{v}_{\text{ab}}-\bar{v}_{\text{auf}}}$ &    Prüfe Gültigkeitsbereich \\
        \midrule
                    1&        0,05 &  0,02 &                                               1,02 &       im Gültigkeitsbereich \\
                    2&        0,04 &  0,02 &                                               1,09 &       im Gültigkeitsbereich \\
                    3&        0,06 &  0,01 &                                               0,48 & nicht im Gültigkeitsbereich \\
                    4&        0,03 &  0,02 &                                               1,00 &       im Gültigkeitsbereich \\
                    5&        0,05 &  0,03 &                                               1,07 &       im Gültigkeitsbereich \\
                    6&        0,03 &  0,02 &                                               1,22 &       im Gültigkeitsbereich \\
                    7&        0,01 &  0,02 &                                               5,74 & nicht im Gültigkeitsbereich \\
                    8&        0,02 &  0,01 &                                               1,09 &       im Gültigkeitsbereich \\
                    9&        0,00 &  0,02 &                                              63,98 & nicht im Gültigkeitsbereich \\
                    10&        0,04 &  0,02 &                                               0,89 &       im Gültigkeitsbereich \\
                    11&        0,05 &  0,03 &                                               1,04 &       im Gültigkeitsbereich \\
                    12&        0,02 &  0,01 &                                               0,78 &       im Gültigkeitsbereich \\
                    13&        0,01 &  0,02 &                                               2,43 & nicht im Gültigkeitsbereich \\
                    14&        0,00 &  0,01 &                                               4,59 & nicht im Gültigkeitsbereich \\
                    15&        0,04 &  0,02 &                                               1,01 &       im Gültigkeitsbereich \\
                    16&        0,01 &  0,03 &                                               3,85 & nicht im Gültigkeitsbereich \\
                    17&        0,05 &  0,02 &                                               0,74 & nicht im Gültigkeitsbereich \\
                    18&        0,02 &  0,02 &                                               1,38 & nicht im Gültigkeitsbereich \\
                    19&        0,03 &  0,02 &                                               1,23 &       im Gültigkeitsbereich \\
                    20&        0,03 &  0,02 &                                               1,09 &       im Gültigkeitsbereich \\
                    21&        0,04 &  0,01 &                                               0,45 & nicht im Gültigkeitsbereich \\
                    22&        -0,01 &  0,01 &                                              -2,00 & nicht im Gültigkeitsbereich \\
        \bottomrule
    \end{tabular}
    \caption{Überprüfung, welche Öltröpfchen im Gültigkeitsbereich liegen.}
    \label{tab:Gültigkeitsbereich}
\end{table}
\clearpage

\subsection{Bestimmung der Ladung und der Radien der Öltröpfchen}

Der Radius und die Ladung der Teilchen wird einmal mit der Cunningham
und einmal ohne Cunningham Korrektur durchgeführt.
Aus \autoref{eqn:q} und \autoref{eqn:Radius} lassen sich nun die Ladungen und der Radius der Tröpfchen bestimmen.
Die Dichte der Luft beträgt $\rho_{\symup{L}}=\SI{1,204}{\kg\per\meter^3}$ und die für Öl beträgt 
$\rho_{\symup{Öl}}=\SI{886}{\kg\per\meter^3}$. Die Viskosität
wird in \autoref{eqn:Radius} und \autoref{eqn:q} durch die Korrektur nach Cunningham in \autoref{eq:Theorie_Cunningham_Viskositaet}
bestimmt. Die elektrische Feldstärke des Plattenkondensators lässt sich aus dem Zusammenhang
$E=\frac{U}{d}$ mit $d=(7,6250 \pm 0,0051)\,\si{\milli\meter}$ bestimmen. Die korrigierte Ladung
ergibt sich aus \autoref{eqn:q_korrigiert}. Die ermittelten Werte sind in \autoref{tab:rq}
abgebildet.

\begin{table}[!ht]
    \centering
    \begin{tabular}{c c c c}
    \hline
        Tröpchen & r in $10^-5$ m  & q in $10^{-19}$ C & $q_{korr}$ in $10^{-19}$ C \\ 
    \hline
        1 & 1,53 & 8,5683 & 8,5686 \\ 
        2 & 1,37 & 4,6705 & 4,6706 \\ 
        4 & 1,19 & 10,7862 & 10,7863 \\ 
        5 & 1,53 & 7,2989 & 7,2992 \\ 
        6 & 1,19 & 8,6035 & 8,6038 \\ 
        8 & 0,97 & 7,6269 & 7,6270 \\ 
        10 & 1,37 & 7,5187 & 7,5190 \\ 
        11 & 1,53 & 10,816 & 10,8163 \\ 
        12 & 0,97 & 5,2623 & 5,2621 \\ 
        15 & 1,37 & 5,9531 & 5,9535 \\ 
        19 & 1,19 & 7,5187 & 7,5189 \\ 
        20 & 1,19 & 10,0977 & 10,0968 \\ 
    \end{tabular}
    \caption{Radien, Ladungen und korrigierte Ladungen der Tröpchen}
    \label{tab:rq}
\end{table}


\subsection{Bestimmung der Elementarladung}
Um die Elementarladung zu bestimmen, werden die berechneten Größen in Bereiche aufgeteilt, die in derselben Größenordnung liegen.
Dann wird die Ladung in den einzelnen Gruppen gemittelt und schließlich ein Mittelwert der Differenzen zwischen zwei "benachbarten" 
Gruppen berechnet.
Für die unkorrigierte Ladung ergibt sich für die Bereiche:
\begin{align*}
    A&=[4.6705] \\
    B&=[5.2953, 5,9531] \\
    C&=[7.2989,7.6269,7.5178,7.5187] \\
    D&=[8.5683,8.6035] \\
    E&=[10.7862,10.8016,10.0977]
\end{align*}
wobei die Ladungen in $10^{-19}$ C angegeben sind.
Durch Mittelwertbildung nach \ref{sec:} ergibt sich
\begin{align*}
    \bar{A}&= 4.6705\\
    \bar{B}&= 5.62\pm 0.33 \\
    \bar{C}&= 7.49 \pm 0.12 \\
    \bar{D}&= 8.586 \pm 0.018 \\
    \bar{E}&= 10.56 \pm 0.33 \\
\end{align*}
Nun werden die Differenzen (B-A),(C-B),(D-C),(E-D) gebildet und gemittelt.
Für die Elementarladung$e_0$ ergibt sich schließlich
\begin{equation*}
    e_0= (1.4728 \pm 0.0821) 10^{-19} C .
\end{equation*}

Die Elementarladung für die mithilfe der Cunningham-Korrektur berechneten Ladungen berechnet sich vollständig analog zu 
\begin{equation*}
    e_{0k}=(1.4741 \pm 0.0829) 10^{-19}C .
\end{equation*}
Dabei wurden die korriegierten Wert für $q_{korr}$ aus Tabelle \ref{tab:} verwendet.
\subsection{Bestimmung der Avogadrokonstante}
Um die Avogadrokonstanten zu bestimmen wird Formel \ref{eq:} verwendet.
Für $F$ wird dabei die Faradaykonstante
\begin{equation*}
    F= 96485.339 \frac{C}{mol}
\end{equation*}
verwendet.
Mit $e_0$ und $e_{0k}$ ergibt sich für die Avogadrokonstanten
\begin{align*}
    N_A&= (6.551 \pm 0.365) \frac{1}{\text{mol}} \\
    N_{Akorr}&= (6.543 \pm 0.368) \frac{1}{\text{mol}}.
\end{align*}

